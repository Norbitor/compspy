\section {Opis projektu}
Celem projektu było utworzenie systemu pozwalającego prowadzącemu zajęcia monitorować komputery w~pracowni laboratoryjnej. Aplikacja do pracy wymaga skonfigurowania serwera aplikacji (IIS z~obsługą ASP.NET), połączonego siecią lokalną z~komputerami w~laboratorium. 

Serwer udostępnia serwis internetowy, do którego prowadzący zajęcia loguje się za pomocą przeglądarki internetowej. Udostępnia on podgląd pulpitów i~informacji o~procesach uruchomionych na aktywnych stanowiskach komputerowych. W~domyślnym widoku dostępny jest podgląd ekranów z~wybranej pracowni, każdy w oddzielnej komórce siatki -- podobnie jak w~systemach monitoringu wizyjnego CCTV. Po wybraniu odpowiedniego stanowiska, program wyświetla podgląd z~pulpitu danego komputera w wyższej jakości wraz z~dodatkowymi danymi. Są to uruchomione aplikacje oraz otwarte karty w~przeglądarkach komputerowych. W~czasie, gdy prowadzący będzie zalogowany na swój panel, aplikacje klienckie zbierają dane i~przesyłają je na serwer. Monitoring nie odbywa się w momencie, gdy żaden prowadzący nie prowadzi monitoringu -- oszczędność zasobów sprzętowych oraz przepustowości sieci lokalnej.

System umożliwia definiowane tzw. ,,czarnej listy'' procesów, a~w~momencie wykrycia, że użytkownik uruchomił jakikolwiek proces z~tej listy, zapisuje we właściwej tabeli informacje o~naruszeniach. Dodatkowo, do każdego naruszenia zapisywany jest zrzut ekranu.
