\section {Wymagania funkcjonalne}
W poniższej sekcji przedstawiono wymagania funkcjonalne projektowanego systemu. Najpierw natomiast muszą zostać przedstawieni aktorzy, których lista i opis znajdują się w poniższej tabeli.


\begin{table}[!ht]
\caption{\label{tab:widgets}Aktorzy systemu}
\begin{tabular}{|c|c|} \hline
Aktor & Opis aktora \\ \hline
Użytkownik   & jest to użytkownik, który obsługuje stanowisko komputerowe\\
             & w pracowni, nie musi on być nawet świadomy, że jest aktorem\\
             & w tym systemie \\ \hline

Administrator   & jest to użytkownik, który ma możliwość zalogowania się \\
                & na stronie, która znajduje się na serwerze i umożliwia \\
                & podgląd stanowisk połączonych z serwerem. \\ \hline

\end{tabular}
\end{table}
\newline\newline\newline
Po zapoznaniu się z krótkimi opisami dotyczących poszczególnych aktorów występujących w systemie można zapoznać się z wymaganiami funkcjonalnymi, które zostały zaprezentowane poniżej.


\begin{center}
\begin{table}[!ht]
\caption{\label{tab:widgets}Lista wymagań funkcjonalnych}
\begin{tabular}{| m{0.5cm} | m{8cm} | m{3cm} | m{2cm} |}
\hline
l.p. 
    & Funkcjonalność 
    & Aktor 
    & Obszar 
\\ \hline
    1.   
    & robienie zrzutów ekranu w określonej częstotliwości i przesyłanie ich do serwera
    & użytkownik
    & aplikacja stanowiska
\\ \hline
    2. 
    &konfiguracja aplikacji stanowiska, wymaga wcześniejszego zalogowania się na konto administratora
    &administrator
    & serwis internetowy
\\ \hline
    3.
    & tworzenie i zarządzanie grupami stanowisk
    & administrator
    & serwis internetowy
\\ \hline
    4. 
    &podgląd wielu stanowisk jednocześnie (wybór jednej lub więcej grup)
    &administrator
    &serwis internetowy
\\ \hline
    5. 
    & podgląd w lepszej jakości obrazu ze stanowisk wraz z listą uruchomionych aplikacji oraz kart z przeglądarek internetowych
    & administrator
    & serwis internetowy
\\ \hline
    6.
    & możliwość włączenia i konfiguracji jednej z dwóch list: white listy lub black listy, które pozwolą na wyświetlanie odpowiednich komunikatów administratorowi połączonemu do serwisu internetowego
    & administrator
    & serwis internetowy
\\ \hline

\end{tabular}\end{table}


\begin{tabular}{| m{0.5cm} | m{8cm} | m{3cm} | m{2cm} |}
\hline 

    7.
    & wyświetlanie komunikatów o użytkownikach, którzy wchodzą na strony spoza white listy lub która znajduje się na black liście w zależności od konfiguracji
    & administrator
    & serwis internetowy
\\ \hline
    8.
    & oglądanie logów serwera oraz poszczególnych stanowisk komputerowych
    & administrator
    & serwis internetowy
\\ \hline
    
\end{tabular}


\end{center}

