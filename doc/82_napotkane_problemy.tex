\lstdefinestyle{sharpc}{language=[Sharp]C, frame=lr, rulecolor=\color{blue!80!black}}


\section{Napotkane problemy}

W poniższym rozdziale opisane zostały najważniejsze problemy z jakimi spotkała się grupa projektowa w trakcie projektowania oraz już samej implementacji.
\newline

\subsection{Tworzenie zrzutów ekranu oraz pobieranie listy procesów aktywnych}
Pierwszym problemem jaki wystąpił, było tworzenie zrzutów ekranu oraz listy procesów na komputerze klienta. To był najłatwiejszy problem z jakim spotkała się grupa. Aby rozwiązać problem wystarczyło wykorzystać dostępne w środowisku .NET funkcje. Samo zrobienie zrzutu to tak na prawdę wykorzystanie dokładnie jednej funkcji.

\newline
\begin{lstlisting}[frame=single,captionpos=b,
    caption={Fragment kodu odpowiedzialny za tworzenie zrzutów ekranu},
    label={lst:kod1},
    style=sharpc]
bmp = new Bitmap(Screen.PrimaryScreen.Bounds.Width,
    Screen.PrimaryScreen.Bounds.Height,
    System.Drawing.Imaging.PixelFormat.Format32bppRgb);
    zrzut = Graphics.FromImage(bmp);
    zrzut.CopyFromScreen(0, 0, 0, 0,
        Screen.PrimaryScreen.Bounds.Size,
        CopyPixelOperation.SourceCopy);
\end{lstlisting}

\begin{lstlisting}[frame=single,captionpos=b,
    caption={Fragment kodu odpowiedzialny za pobranie listy aktywnych procesów},
    label={lst:kod1},
    style=sharpc]
public Process[] listaProcesow;
listaProcesow = Process.GetProcesses();
\end{lstlisting}


\subsection{Uzyskanie adresów kart w przeglądarkach internetowych}
Drugim problemem było pytanie, jak uzyskać adresy stron internetowych. Samo wyświetlenie tytułów było łatwe, ponieważ wykorzystano podobnie jak w poprzednim problemie dostępne funkcje w środowisku. NET. Samo zdobycie tytułów otwartych okien zostało pokazane na przykładzie przeglądarki Chrome na listingu nr 3.
\newline
\begin{lstlisting}[frame=single,captionpos=b,
    caption={Fragment kodu odpowiedzialny za pobranie tytułu otwartej strony w przeglądarce Chrome.},
    label={lst:kod1},
    style=sharpc]
foreach (var p in listaProcesow)
{
    if (Convert.ToString(p.ProcessName) == "chrome")
    {
        if (p.MainWindowTitle != "" ||
            p.MainWindowTitle == " ")
        {
            listaStron.Add("[chrome]");
            listaStron.Add ("Title: " + 
                p.MainWindowTitle);
        }
    }
}
\end{lstlisting}

Powyższy sposób został użyty do pobrania tytułów z przeglądarek: Chrome, Opera, Internet Explorer oraz Microsoft Edge (działanie tej ostatniej nie zostało potwierdzone, ze względu na brak testów w systemie Windows 10). Dzięki wykorzystaniu biblioteki NDde udało się pobrać nie tylko tytuł, ale także adres strony internetowej otwartej w przeglądarce Mozilla Firefox. Ze względu na ograniczony czas pracy nad systemem i wystąpieniem tak wielu problemów, obecnie nie działa to dla innych przeglądarek.

\newline



\begin{lstlisting}[frame=single,captionpos=b,
caption                 ={Fragment kodu odpowiedzialny za pobranie adresu URL oraz tytułu otwartej strony w przeglądarce Mozilla Firefox},
label                   ={lst:kod1},
style                   =sharpc]

using NDde.Client;

public void getURLfirefox()
{
    try
    {
        DdeClient dde = new DdeClient("firefox",
            "WWW_GetWindowInfo");
        dde.Connect();
        string url = dde.Request("URL", int.MaxValue);
        string[] urls = url.Split(new string[] 
            { ",", "\"" }, 
            StringSplitOptions.RemoveEmptyEntries);
        dde.Disconnect();
        
        listaStron.Add("[firefox]");
        listaStron.Add("URL: " + urls[0]);
        listaStron.Add("Title: " + urls[1]);
    }
    catch {}
}
\end{lstlisting}

\subsection{Konwersja zrzutu ekranu}

Następnym problemem, jaki napotkała grupa projektowa, było to to w jaki sposób przesłać zrzut ekranu. Okazało się że najlepszym sposobem będzie konwersja zdjęcia (już po zmianie formatu oraz rozmiaru w zależności od tego w jakiej jakości żąda serwer) do formatu tekstowego za pomocą konwersji do typu Base64. 

\newline
\begin{lstlisting}[frame=single,captionpos=b,
    caption={Fragment kodu odpowiedzialny za konwersję obrazu.},
    label={lst:kod1},
    style=sharpc]
public String getImageBase64(Bitmap img)
{
    Graphics g = Graphics.FromImage(img);
    
    System.IO.MemoryStream stream = 
        new System.IO.MemoryStream();
    img.Save(stream, 
        System.Drawing.Imaging.ImageFormat.Bmp);
    byte[] imageBytes = stream.ToArray();
    
    return Convert.ToBase64String(imageBytes);       
}
\end{lstlisting}

\subsection{Wykorzystanie SignalR}
Po konwersji następuje już budowa samego komunikatu. Następnie następuje serializacja komunikatu i wysłanie go do serwera. Ale nie jest to takie łatwe, ponieważ problemem okazała się kwestia, aby nie przesyłać niepotrzebnie pakietów, gdy żaden użytkownik nie korzysta z serwisu internetowego i nie podgląda stanowisk. Rozwiązaniem okazało się wykorzystanie biblioteki SignalR. Pierwszym listingiem, który został ukazany jest sama konfiguracja uchwytu serwera, który zawiera w sobie hub do obsługi połączenia.

\newpage
\newline
\begin{lstlisting}[frame=single,captionpos=b,
    caption={Fragment kodu - konstruktor klasy ServerHandler},
    label={lst:kod1},
    style=sharpc]
public ServerHandler
    (string hubAddr, ApplicationForm appForm)
 {
    this.hubAddr = hubAddr;
    this.appForm = appForm;
    
    hubConnection = new HubConnection(hubAddr);
    
    computerHub = 
        hubConnection.CreateHubProxy("ComputerHub");
        
    ConfigureRPCHandlers();
}
\end{lstlisting}

Ważnym elementem wykorzystania tej klasy jest metoda ustanawiająca połączenie. Podczas nawiazywania tego połączenia jest przesyłana informacja o identyfikatorze stanowiska oraz sekret.


\begin{lstlisting}[frame=single,captionpos=b,
    caption={Fragment kodu metody nawiązującej połączenie},
    label={lst:kod1},
    style=sharpc]
private void EstablishConnection()
{
    var parameters = new Dictionary<string, string>
    {
        { "stationId", ConfigurationManager.AppSettings
                ["stationDiscr"]},
                
        { "secret", ConfigurationManager.AppSettings
                ["secret"] }
    };
    
    computerHub.Invoke("Connect", 
        ConfigurationManager.AppSettings
        ["stationDiscr"]);
}
\end{lstlisting}

\newpage
Poniżej przedstawiono sposób wysyłania komunikatu dla żądania rozpoczęcia transmisji niskiej jakości. Fragment pochodzi z metody odpowiedzialnej za konfigurację uchwytu RPC.

\begin{lstlisting}[frame=single,captionpos=b,
    caption={Fragment kodu przedstawiający wysyłanie komunikatu dla rozpoczęcia transmisji niskiej jakości.},
    label={lst:kod1},
    style=sharpc]
computerHub.On("StartLowQualityTransmission", () =>
{
    Spy spy = new Spy();
    spy.Aktualizacja();
    var data = spy.serializacja(false);
    computerHub.Invoke("ReceiveData", data);
});
\end{lstlisting}






