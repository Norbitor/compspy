\section{Możliwości rozwoju}

Poniższy rozdział przedstawia możliwe kierunki rozwoju dla systemu. Podzielono te kierunki na dwie kategorie: rozszerzające lub dodające nowe funkcjonalności oraz zwiększające przyjazność obsługi dla użytkownika. Dodatkowo można wspomnieć o możliwości optymalizacji wykorzystania zasobów komputera oraz wykorzystania łącza sieciowego. 

\subsection{Funkcjonalności}
W tym podrozdziale skupiono się na możliwościach rozwoju działających już funkcjonalności oraz takim, które można dodać do systemu.

    Pierwszą rzeczą, jaką można utworzyć jest to utworzenie aplikacji klienckiej dla systemu operacyjnego Linux. Dzięki takiemu rozwiązaniu, system będzie wspierał sale, w których korzysta się z innego systemu operacyjnego niż Windows.
    
    Drugim rozwiązaniem może być połączenie systemu z innymi, w celu np. identyfikacji studenta/ucznia i powiązaniem go ze zdarzeniami związanymi w nim. W takim przypadku można np. wykorzystać do sprawdzenia w późniejszym terminie historii ostrzeżeń studenta bez pamiętania jego stanowiska, przy którym siedział. 
    
    
    Kolejnym rozszerzeniem mogłoby być dodanie większej ilości informacji o procesach i wykorzystaniu komputera przez użytkownika. Może to być informacja u procencie wykorzystania procesora, ilości wykorzystywanej pamięci oraz łącza internetowego. To ostatnie może być także użyte do prowadzenia statystyk związanych z wykorzystywaniem systemu CompSpy w salach laboratoryjnych.
    
    
    Dosyć ważnym rozszerzeniem byłoby utworzenie rozszerzeń do przeglądarek internetowych wykorzystywanych w salach laboratoryjnych, aby była możliwość pobrania wszystkich otwartych kart w przeglądarkach, a nie tylko jednej aktualnie otwartej.
    
\newpage
\subsection{Przyjazność dla użytkownika}
W tej sekcji postanowiono opisać możliwości rozwoju związane z samym interfejsem użytkownika i funkcjonalnościami, które maj poprawić wygodę korzystania z systemu.


Jedną z możliwości polepszenia interfejsu użytkownika jest stworzenie aplikacji okienkowej, która będzie służyć tylko do podglądu stanowisk. Można by taką aplikację udostępniać w danych salach laboratoryjnych na np. jednym stanowisku komputerowym, gdzie dostęp do podglądu nie wymagałby konta w systemie ale byłby powiązany tylko z daną salą. Dzięki takiemu rozwiązaniu nauczyciele nie musieliby się logować do systemu, lecz taki program mógłby działać od początku uruchomienia systemu operacyjnego i w każdym momencie mogliby do niego przejść.

Drugim rozszerzeniem, które zostało uznane przez grupę za ciekawe jest utworzenie wielu wersji językowych systemu, szczególnie wersji angielskiej. Dzięki takiemu rozwiązaniu nauczyciele, którzy są np. z zagranicznych uczelni mogliby też korzystać.