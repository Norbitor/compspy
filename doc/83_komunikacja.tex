\section{Komunikacja}

W poniższym rozdziale zostanie w krótki sposób opisany sposób komunikacji pomiędzy poszczególnymi częściami systemu. Wyróżniamy dwa rodzaje komunikacji: pierwszy to komunikacja użytkownika serwisu internetowego z serwerem tego systemu, drugi to natomiast komunikacja klientów z serwerem. 

\vspace{0.5cm}

Komunikacja pomiędzy serwisem internetowym i jej klientem odbywa się w standardowy sposób dla technologii ASP.NET. Postawiony system na serwerze IIS(Internet Information Services) zawiera pliki napisane z wykorzystaniem HTML, JavaScript oraz C#. Komunikacja odbywa się za pośrednictwem protokołu HTTP.
\vspace{0.5cm}

Komunikacja pomiędzy klientami a serwisem internetowym wykorzystuje bibliotekę SignalR. Wykorzystanie jej jest opisane w rozdziale 12 zatytułowanym "Napotkane problemy". Poniżej natomiast przedstawiona jest budowa komunikatu, który jest wysyłany od klienta do serwera. 

\vspace{0.5cm}
\begin{lstlisting}[frame=single,captionpos=b,
    caption={Fragment kodu przedstawiający budowę klasy Komunikat},
    label={lst:kod1},
    style=sharpc]
[DataContract]
public class Komunikat
{
    [DataMember]
    public String image { get; set; }
    [DataMember]
    public List<String> listaProcesow { get; set; }
    [DataMember]
    public List<String> listaStron { get; set; }

    public Komunikat(String img, Process[] procesy,
        List<String> strony) {
        image = img;
        listaProcesow = new List<String>();
        listaStron = new List<String>();
        foreach (var p in procesy)
            listaProcesow.Add(p.ProcessName);
        foreach (var s in strony)
            listaStron.Add(s);
    }
}
\end{lstlisting}

Podczas tworzenia powyższego komunikatu wprowadzane są
 dane aktualne z listy procesów, listy stron oraz tworzony jest zrzut ekranu. Po tym procesie komunikat jest serializowany. Format samego komunikatu, który jest już przesyłany do serwera jest w formacie XML. W kolejnym listingu przedstawiony jest przykład takiego pliku.
 
 \begin{lstlisting}[frame=single,captionpos=b,
    caption={Przykładowy komunikat przesyłany do serwera w formacie XML},
    label={lst:kod1},
    language=XML]
<Spy.Komunikat 
    xmlns="http://schemas.datacontract.org/2004/07/
        CompSpyAgent"
   xmlns:i="http://www.w3.org/2001/XMLSchema-instance">
  <image>
    ...
    zrzut ekranu jako base64
    ...
  </image>
  <listaProcesow xmlns:a="http://schemas.microsoft.com/2003/
    10/Serialization/Arrays">
    <a:string>svchost</a:string>
    <a:string>notepad++</a:string>
    <a:string>isesrv</a:string>
    <a:string>services</a:string>
    <a:string>winlogon</a:string>
    <a:string>chrome</a:string>
    <a:string>conhost</a:string>
    <a:string>explorer</a:string>
  </listaProcesow>
  <listaStron xmlns:a="http://schemas.microsoft.com/
    2003/10/Serialization/Arrays">
    <a:string>[firefox]</a:string>
    <a:string>URL: https://www.google.pl</a:string>
    <a:string>Title: Google</a:string>
    <a:string>[chrome]</a:string>
    <a:string>Title: Onet.pl - Google Chrome</a:string>
  </listaStron>
</Spy.Komunikat>
\end{lstlisting}