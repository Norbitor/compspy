\section {Wymagania funkcjonalne}
W poniższej sekcji przedstawiono wymagania funkcjonalne projektowanego systemu -- zbiór funkcjonalności, które zestaw aplikacji powinien spełnić. Wymagania dotyczą danych obszarów działania oraz związane są z aktorami (zewnętrzny obiekt wchodzący w interakcję z systemem).

Na początku przedstawiono aktorów. Ich lista wraz z~opisem znajdują się w~Tabeli \ref{tab:actors}.

\begin{table}[!ht]
\caption{Aktorzy systemu}
\label{tab:actors}
\begin{tabular}{| m{2,7cm} | m{11cm} |} \hline
Aktor & Opis aktora \\ \hline
Aplikacja agenta   & Program zainstalowany na stanowisku                        komputerowym. Zbiera dane z maszyny klienta.\\ \hline

Użytkownik zalogowany  & Jest to użytkownik, który ma możliwość                              zalogowania się na stronie, która znajduje się na                         serwerze i umożliwia podgląd\\
                       
                        & stanowisk połączonych z serwerem. \\ \hline

Administrator   & Jest to użytkownik zalogowany, ale oprócz jego \\                                      &funkcjonalności może także zarządzać klientami, grupami \\
                &stanowisk oraz użytkownikami. \\ \hline

\end{tabular}
\end{table}

\newpage

Tabela \ref{tab:functionalreq}. przedstawia zebrane wymagania funkcjonalne projektowanego systemu. W~kolejnych kolumnach opisano żądaną funkcjonalność, aktora (lub aktorów) (według Tabeli \ref{tab:actors}) uczestniczącego w~danej funkcjonalności oraz obszar działania tożsamy z~modułami aplikacji.

\begin{table}[!ht]
\caption{Lista wymagań funkcjonalnych}
\label{tab:functionalreq}
\begin{tabular}{| m{0.5cm} | m{7cm} | m{3cm} | m{2cm} |}
\hline
l.p. 
    & Funkcjonalność 
    & Aktor 
    & Obszar 
\\ \hline
    1.   
    & Wykonywanie zrzutów ekranu w~określonej częstotliwości i~przesyłanie ich do serwera
    & Aplikacja agenta
    & Agent, Serwer
\\ \hline
    2. 
    & Definiowanie stanowisk komputerowych w systemie.
    & Administrator
    & Serwis internetowy
\\ \hline
    3.
    & Tworzenie i zarządzanie grupami stanowisk.
    & Administrator
    & Serwis internetowy
\\ \hline
    4. 
    & Podgląd wielu stanowisk jednocześnie w ramach wybranej grupy.
    & Administrator, Użytkownik zalogowany
    & Serwis internetowy
\\ \hline
    5. 
    & Podgląd obrazu w lepszej jakości ze stanowisk wraz z listą uruchomionych aplikacji oraz kart z przeglądarek internetowych
    & Administrator, Użytkownik zalogowany
    & Serwis internetowy
\\ \hline
    6.
    & Zarządzanie czarną listy procesów, które umożliwią alarmowanie monitorującego o naruszeniu zasad.
    & Administrator
    & Serwis internetowy
\\ \hline
    7.
    & Wysyłanie wiadomości tekstowej do wybranego komputera lub grupy stanowisk.
    & Administrator, Użytkownik zalogowany
    & Serwis internetowy
\\ \hline
    8.
    & Zarządzanie użytkownikami i administratorami.
    & Administrator
    & Serwis internetowy
\\ \hline
    9.
    & Zarządzanie uprawnieniami użytkowników do sal.
    & Administrator
    & Serwis internetowy
\\ \hline
    
\end{tabular}
\end{table}


